\documentclass[11pt]{article}

%%%% Geometry
\usepackage[
    a4paper,
    bindingoffset=1cm,
    left=1cm,
    right=1.5cm,
    top=2cm,
    bottom=2cm,
    footskip=1cm
]{geometry}

%%%% Standard Packages
\usepackage{graphicx}%
\usepackage{multirow}%
\usepackage{amsmath,amssymb,amsfonts}%
\usepackage{amsthm}%
\usepackage{mathrsfs}%
\usepackage[title]{appendix}%
\usepackage{xcolor}%
\usepackage{textcomp}%
\usepackage{manyfoot}%
\usepackage{booktabs}%
\usepackage{algorithm}%
\usepackage{algorithmicx}%
\usepackage[noend]{algpseudocode}%
\usepackage{listings}%

%%%% Additional Packages
\usepackage{mathtools}%
\usepackage{enumitem}%
\usepackage{xcolor}%
\usepackage[utf8]{inputenc}%
\usepackage[T2A]{fontenc}%
\usepackage[unicode, pdftex]{hyperref}%
\usepackage{hyphenat}%
\usepackage[russian, english]{babel}%

%%%% Graphix
\usepackage[font=small]{caption}%
\usepackage[labelformat=empty, position=top]{subcaption}%
\usepackage[export]{adjustbox}%
\usepackage{placeins}%

%%%% Commands
\newcommand{\todo}[1]{{\color{red}[TODO: #1]}}%
\newcommand{\red}[1]{{\color{red}#1}}%
\newcommand{\green}[1]{{\color{green}#1}}%
\newcommand{\bin}[2]{{C_{#1}^{#2}}}%

%%%% Operators
\DeclareMathOperator\PP{\mathbb{P}}%
\DeclareMathOperator\EE{\mathbb{E}}%
\DeclareMathOperator\DD{\mathbb{D}}%
\DeclareMathOperator\II{\mathbb{I}}%
\DeclareMathOperator\FF{\mathcal{F}}%
\DeclareMathOperator\CV{\mathrm{cov}}%

\DeclarePairedDelimiter\abs{\lvert}{\rvert}%
\DeclarePairedDelimiter\norm{\lVert}{\rVert}%
\makeatletter
\let\oldabs\abs
\def\abs{\@ifstar{\oldabs}{\oldabs*}}
\let\oldnorm\norm
\def\norm{\@ifstar{\oldnorm}{\oldnorm*}}
\makeatother

%%%% Environments
\newtheorem{theorem}{Теорема}
\newtheorem{statement}{Утверждение}
\newtheorem{lemma}{Лемма}
\newtheorem{definition}{Определение}
\newtheorem{notification}{Замечание}
\newtheorem{question}{Вопрос}
\newtheorem{example}{Пример}
\newtheorem{problem}{Задача}

\title{Условная вероятность. Формула Байеса. Независимость.} 
\date{\vspace{-1cm}}

\begin{document}
\maketitle

\section*{Введение}
\begin{itemize}
    \item
    представление о вероятности события зависит от нашей информированности: пример с тузом пик в руке моего соперника;
    \item
    условная вероятность в классическом случае: обоснование формального определения условной вероятности.
\end{itemize}

\section*{Условная вероятность}
\begin{itemize}
    \item
    общее определение условной вероятности: $\PP(A \left| B\right.) = \PP(AB) / \PP(B)$;
    \item
    случай с последовательностью условий: последовательно произошло два события $B$ и $C$, как пересчитать вероятность события $A$?
    \[
        P(A \left| BC\right.);
    \]
    \item
        вероятность того, что одновременно произошо несколько событий (\textbf{теорема умножения}):
    \[
        \PP\left(A_n \ldots A_1\right) = \PP\left(A_n\left|A_{n - 1} \ldots A_1\right.\right)\cdots\PP\left(A_2\left|A_1\right.\right) \PP\left(A_1\right);
    \]
    \item
    такой подход удобнее, мы легко можем найти условные вероятности (потому что легко работаем в новой реальности, заданной условием), а вот вероятности пересечения нам получить сложнее.
\end{itemize}

\section*{Формула полной вероятности и формула Байеса}
\begin{itemize}
    \item
    определение разбиения вероятностного пространства: $\Omega = \sum_{i = 1}^N B_i$, где $\{B_i\}_{i = 1}^N$ не пересекаются;
    \item
        как найти вероятность события, если мы знаем его условные вероятности на всем разбиении и вероятности разбиения (\textbf{формула полной вероятности}):
    \[
        \PP(A) = \sum_{i = 1}^N \PP\left(A\left|B_i\right.\right) \PP(B_i);
    \]
    \item пусть произошло событие $A$, найдем вероятность того, что оно произошло при условии $B_1$, взятого из набора разбиения $\Omega = \sum_{i = 1}^N B_i$, (\textbf{формула Байеса}):
    \[
        \PP(B_1 \left| A\right.) = \frac{\PP(A \left| B_1\right.) \PP(B_1)}{\sum_{i = 1}^N \PP(A \left| B_i\right.) \PP(B_i)};
    \]
\end{itemize}

\section*{Независимость}
\begin{itemize}
    \item
    независимость двух событий определяется через условную вероятность или через правило произведение: события $A$ и $B$ независимы, если
    \[
        \PP(A \left| B\right.) = \PP(A),\ \text{или}\ \PP(B) = 0\ \Longleftrightarrow\ \PP(AB) = \PP(A) \PP(B);
    \]
    \item
    совместная независимость набора событий $A_1, \ldots, A_n$:
    для любого набора $1 \leq i_1 \leq \ldots \leq i_k \leq n$ должно быть выполнено:
    \[
        \PP\left(A_{i_1} \cdot \ldots \cdot A_{i_k}\right) = \PP\left(A_{i_1}\right) \cdot \ldots \cdot \PP\left(A_{i_k}\right);
    \]
    \item
    обычно независимость явно проговаривается в условиях задачи;
    \item
    пример событий независимых попарно но не в совокупности: 
    ``при первом броске выпал орел'', ``при втором броске выпал орел'' и ``ровно при одном из бросков выпал орел''.
\end{itemize}

\section*{Задачи}
\begin{problem}
    Четыре человека А, Б, В, Г становятся в очередь в случайном порядке.
    Найдите:\\
    а) условную вероятность того, что A первый, если Б последний;\\
    б) условную вероятность того, что A первый, если А не последний;\\
    в) условную вероятность того, что А первый, если Б не последний.
    % г) условную вероятность того, что А первый, если Б стоит в очереди позже А;
\end{problem}

\begin{problem}
    В ящике $N$ шаров, $K$ из которых белые, а остальные черные.
    Вынимают последовательно $n$ шаров без возвращения.
    Какова вероятность, что $j$–ый шар в выборке белый, если известно, что в выборке ровно $k$ белых шаров?
\end{problem}

\begin{problem}
    В ящике $12$ белых шаров и $8$ черных.
    Из ящика извлекли наудачу $3$ шарика, причем известно, что один из них оказался белым.
    Какова вероятность, что белых шаров в выборке больше, чем черных?
\end{problem}

% \begin{problem}
%     Чему равна вероятность того, что при трех вытягиваниях без возвращения из урны с 8 черными, 5 белыми и 3 красными шарами, мы вытянем черный, белый и снова черный шары.
% \end{problem}

\begin{problem}
    Треть выпускников ``ФБиБ'' и четверть выпускников ``Биофака'' знают теорию вероятностей на ``отлично''.
    На выпуске встретились студенты этих факультетов, причем ``ФБиБ'' было в $ 3 / 2$ раза меньше.
    Произвольно выбранный студент знает курс на ``отлично''.
    Отгадайте, какого факультета он выпускник?
\end{problem}

\begin{problem}
    Независимы ли события ``первая карта -- туз'' и ``вторая карта -- туз'' при вытаскивании двух карт без возвращения из колоды в $36$ карт?
\end{problem}

% \begin{problem}
%     Я купил два лотерейных билета ``Русское лото''.
%     Независимы ли события ``первый билет выигрышный'' и ``второй билет выигрышный''?
% \end{problem}

\end{document}
