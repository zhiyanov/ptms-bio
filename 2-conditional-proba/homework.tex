\documentclass[11pt]{article}

%%%% Geometry
\usepackage[
    a4paper,
    bindingoffset=1cm,
    left=1cm,
    right=1.5cm,
    top=2cm,
    bottom=2cm,
    footskip=1cm
]{geometry}

%%%% Standard Packages
\usepackage{graphicx}%
\usepackage{multirow}%
\usepackage{amsmath,amssymb,amsfonts}%
\usepackage{amsthm}%
\usepackage{mathrsfs}%
\usepackage[title]{appendix}%
\usepackage{xcolor}%
\usepackage{textcomp}%
\usepackage{manyfoot}%
\usepackage{booktabs}%
\usepackage{algorithm}%
\usepackage{algorithmicx}%
\usepackage[noend]{algpseudocode}%
\usepackage{listings}%

%%%% Additional Packages
\usepackage{mathtools}%
\usepackage{enumitem}%
\usepackage{xcolor}%
\usepackage[utf8]{inputenc}%
\usepackage[T2A]{fontenc}%
\usepackage[unicode, pdftex]{hyperref}%
\usepackage{hyphenat}%
\usepackage[russian, english]{babel}%

%%%% Graphix
\usepackage[font=small]{caption}%
\usepackage[labelformat=empty, position=top]{subcaption}%
\usepackage[export]{adjustbox}%
\usepackage{placeins}%

%%%% Commands
\newcommand{\todo}[1]{{\color{red}[TODO: #1]}}%
\newcommand{\red}[1]{{\color{red}#1}}%
\newcommand{\green}[1]{{\color{green}#1}}%
\newcommand{\bin}[2]{{C_{#1}^{#2}}}%

%%%% Operators
\DeclareMathOperator\PP{\mathbb{P}}%
\DeclareMathOperator\EE{\mathbb{E}}%
\DeclareMathOperator\DD{\mathbb{D}}%
\DeclareMathOperator\II{\mathbb{I}}%
\DeclareMathOperator\FF{\mathcal{F}}%

\DeclarePairedDelimiter\abs{\lvert}{\rvert}%
\DeclarePairedDelimiter\norm{\lVert}{\rVert}%
\makeatletter
\let\oldabs\abs
\def\abs{\@ifstar{\oldabs}{\oldabs*}}
\let\oldnorm\norm
\def\norm{\@ifstar{\oldnorm}{\oldnorm*}}
\makeatother

%%%% Environments
\newtheorem{theorem}{Теорема}
\newtheorem{statement}{Утверждение}
\newtheorem{lemma}{Лемма}
\newtheorem{definition}{Определение}
\newtheorem{notification}{Замечание}
\newtheorem{question}{Вопрос}
\newtheorem{example}{Пример}
\newtheorem{problem}{Задача}

\title{Условная вероятность. Формула Байеса. Независимость.} 
\date{\vspace{-1cm}}

\begin{document}
\maketitle

\section*{Домашнее задание}
\begin{problem}
    Четыре человека А, Б, В, Г становятся в очередь в случайном порядке.
    Найдите условную вероятность того, что А первый, если Б стоит в очереди позже А.
\end{problem}

\begin{problem}
    В шестизарядном револьвере только два патрона, вставленные подряд.
    При попытке нажать на спусковой крючок выстрела не случилось.
    Какие шансы на то, что он случится при следующем нажатии на курок (без раскручивания барабана).
\end{problem}

\begin{problem}
    Группу из $3 n$ девушек и $3 n$ юношей разделили на три команды по $2 n$ человек в каждой.
    Какова вероятность, что в каждой команде ровно $n$ юношей и девушек, если такая информация уже известна о первой команде?
\end{problem}

\begin{problem}
    Чему равна вероятность того, что при трех вытягиваниях без возвращения из урны с 8 черными, 5 белыми и 3 красными шарами, мы вытянем черный, белый и снова черный шары.
\end{problem}

\begin{problem}
    Три стрелка стреляют по мишени.
    Вероятность попасть в мишень у стрелков независимы и равны $0.2$, $0.4$ и $0.6$.
    Какова вероятность, что второй стрелок в мишень попал, если в мишени оказалось две пробоины?
\end{problem}

\end{document}