\documentclass[11pt]{article}

%%%% Geometry
\usepackage[
    a4paper,
    bindingoffset=1cm,
    left=1cm,
    right=1.5cm,
    top=2cm,
    bottom=2cm,
    footskip=1cm
]{geometry}

%%%% Standard Packages
\usepackage{graphicx}%
\usepackage{multirow}%
\usepackage{amsmath,amssymb,amsfonts}%
\usepackage{amsthm}%
\usepackage{mathrsfs}%
\usepackage[title]{appendix}%
\usepackage{xcolor}%
\usepackage{textcomp}%
\usepackage{manyfoot}%
\usepackage{booktabs}%
\usepackage{algorithm}%
\usepackage{algorithmicx}%
\usepackage[noend]{algpseudocode}%
\usepackage{listings}%

%%%% Additional Packages
\usepackage{mathtools}%
\usepackage{enumitem}%
\usepackage{xcolor}%
\usepackage[utf8]{inputenc}%
\usepackage[T2A]{fontenc}%
\usepackage[unicode, pdftex]{hyperref}%
\usepackage{hyphenat}%
\usepackage[russian, english]{babel}%

%%%% Graphix
\usepackage[font=small]{caption}%
\usepackage[labelformat=empty, position=top]{subcaption}%
\usepackage[export]{adjustbox}%
\usepackage{placeins}%

%%%% Commands
\newcommand{\todo}[1]{{\color{red}[TODO: #1]}}%
\newcommand{\red}[1]{{\color{red}#1}}%
\newcommand{\green}[1]{{\color{green}#1}}%
\newcommand{\bin}[2]{{C_{#1}^{#2}}}%

%%%% Operators
\DeclareMathOperator\PP{\mathbb{P}}%
\DeclareMathOperator\EE{\mathbb{E}}%
\DeclareMathOperator\DD{\mathbb{D}}%
\DeclareMathOperator\II{\mathbb{I}}%
\DeclareMathOperator\FF{\mathcal{F}}%
\DeclareMathOperator\CV{\mathrm{cov}}%

\DeclarePairedDelimiter\abs{\lvert}{\rvert}%
\DeclarePairedDelimiter\norm{\lVert}{\rVert}%
\makeatletter
\let\oldabs\abs
\def\abs{\@ifstar{\oldabs}{\oldabs*}}
\let\oldnorm\norm
\def\norm{\@ifstar{\oldnorm}{\oldnorm*}}
\makeatother

%%%% Environments
\newtheorem{theorem}{Теорема}
\newtheorem{statement}{Утверждение}
\newtheorem{lemma}{Лемма}
\newtheorem{definition}{Определение}
\newtheorem{notification}{Замечание}
\newtheorem{question}{Вопрос}
\newtheorem{example}{Пример}
\newtheorem{problem}{Задача}

\title{Условная вероятность. Формула Байеса. Независимость.} 
\date{\vspace{-1cm}}

\begin{document}
\maketitle

\section*{Банк задач}
\begin{problem}
    Четыре человека А, Б, В, Г становятся в очередь в случайном порядке.
    Найдите условную вероятность того, что А первый, если Б стоит в очереди позже А.
\end{problem}

\begin{problem}
    Брошено $2$ кубика.
    Найти условную вероятность того, что выпали $2$ пятерки, если известно, что сумма выпавших очков делится на $5$.
\end{problem}

\begin{problem}
    Группу из $3 n$ девушек и $3 n$ юношей разделили на три команды по $2 n$ человек в каждой.
    Какова вероятность, что в каждой команде ровно $n$ юношей и девушек, если такая информация уже известна о первой команде?
\end{problem}

\begin{problem}
    Пусть события $A$, $B$, $C$ и $D$ попарно независимы.
    Можно ли тогда утверждать, что независимы события $AB$ и $CD$?
    Если да, то докажите это.
    Если нет, то привидите пример событий, когда это не так.
\end{problem}

\begin{problem}
    Три стрелка стреляют по мишени.
    Вероятность попасть в мишень у стрелков независимы и равны $0.2$, $0.4$ и $0.6$.
    Какова вероятность, что второй стрелок в мишень попал, если в мишени оказалось две пробоины?
\end{problem}

\begin{problem}
    Из колоды карт ($52$ шт) крупье достает две и сообщает, что одна из них туз.
    Найти вероятность того, что а) вторая карта -- тоже туз, б) третья карта, извлеченная из колоды, -- туз?
\end{problem}

\begin{problem}
    Подсудимому предлагают разложить десять белых и десять чёрных шаров по двум одинаковым коробкам (надо использовать все шары; в каждой коробке должен быть хотя бы один шар).
    После этого судья равновероятно выбирает случайную коробку, а затем равновероятно шар из этой коробки.
    Если шар чёрный, то подсудимого казнят, если белый — отпускают.
    Как нужно разложить шары, чтобы вероятность выжить была максимальной (подбор можно осуществить на компьютере)?
\end{problem}

\begin{problem}
    Брошено случайное число N кубиков, $\PP(N = k) = 2^{-k}$.
    Найти вероятность того, что:\\
    а) $N = 2$, если все выпавшие числа различны,\\
    б) cумма выпавших числе $S = 5$, если $N$ нечетно.
\end{problem}

\begin{problem}
    После экскурсии самый уставший турист ушел в автобус первым и сел на случайное место.
    После этого остальные заходили в автобус по одному и садились на свое место, если оно свободно, и на любое
другое в противном случае.
    Какова вероятность того, что гид, зашедший последним, сможет сесть на свое место с микрофоном (всего в автобусе $n$ мест, в группе $n - 1$ турист и один гид)?
\end{problem}

\begin{problem}
    \textbf{Закон Харди-Вайнберга.}
    В некоторой стране у $20\%$ жителей глаза светлые, а у остальных — темные.
    Считая ген светлых глаз рецессивным, а ген темных глаз доминантным, найдите процент светлоглазых жителей в следующем поколении.
    Что произойдет в третьем, четвертом и т.д. поколении?
\end{problem}

\begin{problem}
    Четыре человека А, Б, В, Г становятся в очередь в случайном порядке.
    Найдите:\\
    а) условную вероятность того, что A первый, если Б последний;\\
    б) условную вероятность того, что A первый, если А не последний;\\
    в) условную вероятность того, что А первый, если Б не последний.
    % г) условную вероятность того, что А первый, если Б стоит в очереди позже А;
\end{problem}

\begin{problem}
    В шестизарядном револьвере только два патрона, вставленные подряд.
    При попытке нажать на спусковой крючок выстрела не случилось.
    Какие шансы на то, что он случится при следующем нажатии на курок (без раскручивания барабана).
\end{problem}

\begin{problem}
    В ящике $N$ шаров, $K$ из которых белые, а остальные черные.
    Вынимают последовательно $n$ шаров без возвращения.
    Какова вероятность, что $j$–ый шар в выборке белый, если известно, что в выборке ровно $k$ белых шаров?
\end{problem}

\begin{problem}
    В ящике $12$ белых шаров и $8$ черных.
    Из ящика извлекли наудачу $3$ шарика, причем известно, что один из них оказался белым.
    Какова вероятность, что белых шаров в выборке больше, чем черных?
\end{problem}

\begin{problem}
    Чему равна вероятность того, что при трех вытягиваниях без возвращения из урны с 8 черными, 5 белыми и 3 красными шарами, мы вытянем черный, белый и снова черный шары.
\end{problem}

\begin{problem}
    Треть выпускников ``ФБиБ'' и четверть выпускников ``Биофака'' знают теорию вероятностей на ``отлично''.
    На выпуске встретились студенты этих факультетов, причем ``ФБиБ'' было в $ 3 / 2$ раза меньше.
    Произвольно выбранный студент знает курс на ``отлично''.
    Отгадайте, какого факультета он выпускник?
\end{problem}

\begin{problem}
    Независимы ли события ``первая карта -- туз'' и ``вторая карта -- туз'' при вытаскивании двух карт без возвращения из колоды в $36$ карт?\\
    Я купил два лотерейных билета ``Русское лото''.
    Независимы ли события ``первый билет выигрышный'' и ``второй билет выигрышный''?
\end{problem}

\begin{problem}
    У Анны есть красный, синий и зеленый чайные наборы (чашка, блюдце, тарелка, салфетка).
    Каждое утро Анна равновероятно выбирает цвет сервиза для чаепития.
    Ее маленькая дочь, накрывая на стол, выбирает для каждого предмета (независимо от остальных) нужный цвет с вероятностью $2 / 3$, любой из двух оставшихся – с $1 / 6$.
    Какова вероятность того, что сегодня нужно было поставить синий сервиз, если на столе стоят синие чашка и блюдце, тарелка оказалась красной, а салфетка зеленая?
\end{problem}

\end{document}