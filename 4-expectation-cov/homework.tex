\documentclass[11pt]{article}

%%%% Geometry
\usepackage[
    a4paper,
    bindingoffset=1cm,
    left=1cm,
    right=1.5cm,
    top=2cm,
    bottom=2cm,
    footskip=1cm
]{geometry}

%%%% Standard Packages
\usepackage{graphicx}%
\usepackage{multirow}%
\usepackage{amsmath,amssymb,amsfonts}%
\usepackage{amsthm}%
\usepackage{mathrsfs}%
\usepackage[title]{appendix}%
\usepackage{xcolor}%
\usepackage{textcomp}%
\usepackage{manyfoot}%
\usepackage{booktabs}%
\usepackage{algorithm}%
\usepackage{algorithmicx}%
\usepackage[noend]{algpseudocode}%
\usepackage{listings}%

%%%% Additional Packages
\usepackage{mathtools}%
\usepackage{enumitem}%
\usepackage{xcolor}%
\usepackage[utf8]{inputenc}%
\usepackage[T2A]{fontenc}%
\usepackage[unicode, pdftex]{hyperref}%
\usepackage{hyphenat}%
\usepackage[russian, english]{babel}%

%%%% Graphix
\usepackage[font=small]{caption}%
\usepackage[labelformat=empty, position=top]{subcaption}%
\usepackage[export]{adjustbox}%
\usepackage{placeins}%

%%%% Commands
\newcommand{\todo}[1]{{\color{red}[TODO: #1]}}%
\newcommand{\red}[1]{{\color{red}#1}}%
\newcommand{\green}[1]{{\color{green}#1}}%
\newcommand{\bin}[2]{{C_{#1}^{#2}}}%

%%%% Operators
\DeclareMathOperator\PP{\mathbb{P}}%
\DeclareMathOperator\EE{\mathbb{E}}%
\DeclareMathOperator\DD{\mathbb{D}}%
\DeclareMathOperator\II{\mathbb{I}}%
\DeclareMathOperator\FF{\mathcal{F}}%

\DeclarePairedDelimiter\abs{\lvert}{\rvert}%
\DeclarePairedDelimiter\norm{\lVert}{\rVert}%
\makeatletter
\let\oldabs\abs
\def\abs{\@ifstar{\oldabs}{\oldabs*}}
\let\oldnorm\norm
\def\norm{\@ifstar{\oldnorm}{\oldnorm*}}
\makeatother

%%%% Environments
\newtheorem{theorem}{Теорема}
\newtheorem{statement}{Утверждение}
\newtheorem{lemma}{Лемма}
\newtheorem{definition}{Определение}
\newtheorem{notification}{Замечание}
\newtheorem{question}{Вопрос}
\newtheorem{example}{Пример}
\newtheorem{problem}{Задача}

\title{Математическое ожидание, дисперсия, ковариация}
\date{\vspace{-1cm}}

\begin{document}
\maketitle

\section*{Домашнее задание}
% \begin{problem}
%     % 2
%     Студент выучил $m$ вопросов из $n$ (здесь $m < n$).
%     Экзаменатор спросил $k$ вопросов (предположим, что $k \leq m$ и $k \leq n - m$).
%     Случайная величина $X$ -- число неизвестных студенту вопросов, среди заданных.
%     Найдите $\EE X$.
% \end{problem}

\begin{problem}
    % 3
    Игрок подбрасывает пару игральных кубиков до тех пор, пока при очередном подбрасывании не выпадет комбинация ``шесть, шесть''.
    Случайная величина $X$ -- число подбрасываний.
    Найдите $\EE X$ и $\DD X$.\\
    \textbf{Подсказка:} почленно продифференцируйте сумму ряда
    $
        f(p) = \sum_{i = 0}^{\infty} p (1 - p)^k.
    $
    Сравните получившийся результат с формулой для подсчета маетматического ожидания.
\end{problem}

\begin{problem}
    % 7
    Найдите математическое ожидание числа повторений грани при подбрасывании четырех игральных костей.
\end{problem}

% \begin{problem}
%     % 12
%     Приведите пример слчайной величины, у которой математическое ожидание не определено (сумма в определении математического ожидания должна быть бесконечной; эта сумма, то есть ряд, должна расходиться).
%     Приведите пример случайной величины, у которой математическое ожидание определено и конечно, а дисперсия бесконечна.
% \end{problem}

\begin{problem}
    Пусть случайная величина $X \sim R\{-2, 1, 0, 1, 2\}$ имеет равномерное распределение.
    Рассмотрим случайную величину $Y = X^2 - 1$.
    Найдите $\CV(X, Y)$.
    Верно ли, что $X$ и $Y$ независимы?
\end{problem}

\begin{problem}
    % 14
    Пусть $\CV(X, Y) = 1$.
    Найдите $\CV(2X + 3Y, Y)$.
\end{problem}

% \begin{problem}
%     % 19
%     По окружности расставили $2$ единицы и $N - 2$ нолика в случайном порядке.
%     Случайная величина $X$ -- число ноликов между парой единиц (минимальное среди двух дуг).
%     Найдите $\EE X$. 
% \end{problem}

\begin{problem}
    % 11
    Совместное распределение пары $(X, Y)$ задано таблицей
    \begin{center}
    \begin{tabular}{c|c|c|c}
        X / Y & 0 & 1 & 2 \\
        \hline
        -1 & 0.2 & 0.05 & 0.15 \\
        \hline
        0 & 0.05 & 0.1 & 0.05 \\
        \hline
        1 & 0.05 & 0.15 & 0.2 \\
    \end{tabular}
    \end{center}
    Найдит $\DD X^2,\ \DD Y^2,\ \EE XY,\ \EE (X + Y)^2$.
    Являются ли $X$ и $Y$ независимыми?
\end{problem}
\end{document}


