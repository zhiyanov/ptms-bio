\documentclass[11pt]{article}

%%%% Geometry
\usepackage[
    a4paper,
    bindingoffset=1cm,
    left=1cm,
    right=1.5cm,
    top=2cm,
    bottom=2cm,
    footskip=1cm
]{geometry}

%%%% Standard Packages
\usepackage{graphicx}%
\usepackage{multirow}%
\usepackage{amsmath,amssymb,amsfonts}%
\usepackage{amsthm}%
\usepackage{mathrsfs}%
\usepackage[title]{appendix}%
\usepackage{xcolor}%
\usepackage{textcomp}%
\usepackage{manyfoot}%
\usepackage{booktabs}%
\usepackage{algorithm}%
\usepackage{algorithmicx}%
\usepackage[noend]{algpseudocode}%
\usepackage{listings}%

%%%% Additional Packages
\usepackage{mathtools}%
\usepackage{enumitem}%
\usepackage{xcolor}%
\usepackage[utf8]{inputenc}%
\usepackage[T2A]{fontenc}%
\usepackage[unicode, pdftex]{hyperref}%
\usepackage{hyphenat}%
\usepackage[russian, english]{babel}%

%%%% Graphix
\usepackage[font=small]{caption}%
\usepackage[labelformat=empty, position=top]{subcaption}%
\usepackage[export]{adjustbox}%
\usepackage{placeins}%

%%%% Commands
\newcommand{\todo}[1]{{\color{red}[TODO: #1]}}%
\newcommand{\red}[1]{{\color{red}#1}}%
\newcommand{\green}[1]{{\color{green}#1}}%
\newcommand{\bin}[2]{{C_{#1}^{#2}}}%

%%%% Operators
\DeclareMathOperator\PP{\mathbb{P}}%
\DeclareMathOperator\EE{\mathbb{E}}%
\DeclareMathOperator\DD{\mathbb{D}}%
\DeclareMathOperator\II{\mathbb{I}}%
\DeclareMathOperator\FF{\mathcal{F}}%
\DeclareMathOperator\CV{\mathrm{cov}}%

\DeclarePairedDelimiter\abs{\lvert}{\rvert}%
\DeclarePairedDelimiter\norm{\lVert}{\rVert}%
\makeatletter
\let\oldabs\abs
\def\abs{\@ifstar{\oldabs}{\oldabs*}}
\let\oldnorm\norm
\def\norm{\@ifstar{\oldnorm}{\oldnorm*}}
\makeatother

%%%% Environments
\newtheorem{theorem}{Теорема}
\newtheorem{statement}{Утверждение}
\newtheorem{lemma}{Лемма}
\newtheorem{definition}{Определение}
\newtheorem{notification}{Замечание}
\newtheorem{question}{Вопрос}
\newtheorem{example}{Пример}
\newtheorem{problem}{Задача}

\title{Математическое ожидание, дисперсия, ковариация}
\date{\vspace{-1cm}}

\begin{document}
\maketitle

\section*{Определение математического ожидания}
\begin{itemize}
    \item
    значение случайной величины нам заранее не известно, тем не менее нам интересно понять, как она устроена ``в среднем'';
    \item
    \textbf{определение:} математическим ожиданием случайной величины $X(\omega)$ назовем
    \[
        \EE X = \sum_{k = 1}^{\infty} X(\omega_k) \PP(\omega_k),
    \]
    где ряд предполагается сходящимся абсолютно (иначе мы не можем менять порядок слагаемых);
    \item
    \textbf{пример:} мaтематическое ожидание числа, выпавшего при подбрасывании кубика;
    \item
    когда речь идет о случайной величине и ее численных характеристиках, нам достаточно знать лишь ее распределение (нам не важно как устроено вероятностное пространство, на котором случайная величина определена);
    \item
    как же переписать математическое ожидание, пользуясь лишь ее распределением?
    \[
        \EE X = \sum_{k} x_k \PP_X(k),
    \] 
    где $\PP_X(\cdot)$ -- функция масс случайной величины $X$ (переформуллировка справедлива, если ряд в правой части сходится абсолютно);
    \item
    переформуллировка математического ожидания в терминах $X^{+}$ и $X^{-}$.
\end{itemize}

\section*{Математическое ожидание функции от случайной величины}
\begin{itemize}
    \item
    пусть имеем набор случайных величин $X_1, X_2, \ldots, X_m$, принимающих значения\\
    $x_{1, 1}, \ldots, x_{1, n_1};\ \ldots;\ x_{m, 1}, \ldots, x_{m, n_m},$ соответственно;
    пусть также $g: \mathbb{R}^m \rightarrow \mathbb{R}$ -- измеримое отображение, тогда $g(X_1, \ldots, X_m)$ -- случайная величина;
    \item
    как искать ее математическое ожидание?
    \item
    \textbf{первый способ:} найти распределение этой случайной величины (с какой вероятностью она принимает конкретные значения) -- может занять много усилий;
    \item
    \textbf{второй способ:} воспользоваться формулой
    \[
        \EE g(X_1, \ldots, X_m) = \sum_{i_1, \ldots, i_m} g(x_{1, i_1}, \ldots, x_{m, i_m}) \PP_{\overrightarrow{X}}(x_{1, i_1}, \ldots, x_{m, i_m}),
    \]
    если ряд сходится абсолютно (здесь $\PP_{\overrightarrow{X}}(\cdot, \ldots, \cdot)$ -- функция масс случайного вектора $\overrightarrow{X} = (X_1, \ldots, X_m)$);
    \item
    \textbf{третий способ:} пусть $X$ -- целочисленная неотрицательная случайная величина, тогда ее \textbf{производящей функцией} назовем
    \[
        \phi_X(s) = \sum_{k = 0}^{\infty} \PP_X(k) s^k,
    \]
    где $\PP_X(\cdot)$ -- функция масс;
    \item
    заметим, что \textbf{степенной} ряд в определении $\phi_X(s)$ сходится в точке $s = 1$, а значит cходится и в интервале $(-1, 1)$;
    \item
    более того, внутри этого интервала сумма ряда $\phi_X(s)$ дифференцируема и ее производная равна
    \[
        \phi'_X(s) = \sum_{k = 1}^{\infty} k \PP_X(k) s^{k - 1};
    \]
    \item
    если ряд в правой части сходится в точке $s = 1$ (то есть математическое ожидание существует), то у $\phi_X(s)$ в точке $s = 1$ существует производная слева, и она равна 
    \[
        \phi_X'(1) = \sum_{k = 1}^{\infty} k \PP_X(k) = \EE X;
    \]
    \item
    таким образом, если математическое ожидание существует, то его можно найти по алгоритму:
    \begin{itemize}
        \item
        найти производящую функцию $\phi_X(s)$;
        \item
        найти ее производную;
        \item
        подставить в результат точку $s = 1$.
    \end{itemize}
\end{itemize}

\section*{Свойства математического ожидания}
\begin{itemize}
    \item
    математическое ожидание константы: $\EE c = c$, где $c$ -- число;
    \item
    математическое ожидание индикатора события: $\EE \II_A = \PP(A)$, где $A$ -- некоторое событие;
    \item
    линейность: для любых чисел $a$ и $b$ и случайных величин $X,\ Y$ с конечным математическим ожиданием
    \[
        \EE\left(a X + b Y\right) = a \EE X + b \EE Y;
    \]
    \item
    мультипликативность: если случайные величины $X$ и $Y$ независимы и имеют конечное математическое ожидание, то 
    \[
        \EE X Y = \EE X \EE Y.
    \]
\end{itemize}

\section*{Неравенства для математических ожиданий (дополнительно)}
\begin{itemize}
    \item
    неравенство Йенсена: пусть $f$ -- выпуклая функция, тогда $\EE f(X) \geq f( \EE X),$ если математические ожидания существуют;
    \item
    неравенство Ляпунова: пусть $p > q > 0,$ $\EE \abs{X}^{p} < \infty,$ тогда 
    \[
        \left( \EE \abs{X}^q \right)^{ \frac{1}{q} } \leq \left( \EE \abs{X}^p \right)^{ \frac{1}{p} };
    \]
    \item
    неравенство Гельдера: пусть $p, q > 1$, $1 / p + 1 / q = 1$ и математические ожидания $\EE \abs{X}^{q} < \infty$, $\EE \abs{X}^{p} < \infty$, тогда
    \[
        \EE |XY| \leq \left(\EE \abs{X}^{q} \right)^{ \frac{1}{q} } \left( \EE \abs{X}^{p} \right)^{ \frac{1}{p} };
    \]
    \item
    неравенство Коши-Буняковского: частный случай неравенства Гельдера для $p = q = 2$.
\end{itemize}

\section*{Дисперсия и ковариация}
\begin{itemize}
    \item
    математическое ожидание описывает среднее;
    логично рассмотреть также дисперсию -- величину, характеризующую разброс вокруг среднего;
    \item
    \textbf{определение:} дисперсией случайной величины $X$ называется число
    \[
        \DD X = \EE\left(X - \EE X\right)^2 \geq 0;
    \]
    \item
    более удобная формула:
    \[
        \DD X = \EE X^2 - \left(\EE X\right)^2;
    \]
    \item
    еще один способ подсчета: если известна производящая функция $\phi_X(s)$ и дисперсия существует, то
    \[
        \DD X = \phi''_X(1) + \phi'_X(1) - \left(\phi'_X(1)\right)^2.
    \]
    \item
    \textbf{определение:} ковариацией случайных величин $X, Y$ называют число 
    \[
        \CV\left(X, Y\right) = \EE\left(X - \EE X\right)\left(Y - \EE Y\right).
    \]
\end{itemize}

\section*{Свойства дисперсии и ковариации}
\begin{itemize}
    \item
    $
        \DD X = \CV\left(X, X\right);
    $
    \item
    $
        \CV\left(X, Y\right) = \CV\left(Y, X\right);
    $
    \item
    $
        \CV\left(X, Y + c\right) = \CV\left(X, Y\right),\ \DD\left(X + c\right) = \DD X;
    $
    \item
    если $X, Y$ независимы, то
    $
        \CV(X, Y) = 0;
    $
    \item
    $
        \DD \left(a X\right) = a^2 \DD X;
    $
    \item
    $
        \DD\left(X_1 + \ldots + X_n\right) = \sum_{i = 1}^n \DD X_i + 2 \sum_{i < j} \CV\left(X_i, X_j\right).
    $
\end{itemize}

\section*{Задачи}
\begin{problem}
    % 1
    При подбрасывании коробка со спичками вероятность того, что он выпадет этикеткой кверху (этикетка наклеена только на одну грань коробка) равна $p$ (назовем такое событие ``успехом'').
    Коробок подбросили $n$ раз.
    Случайная величина $X$ -- число ``успехов''.
    Найдите $\EE X$ и $\DD X$.
\end{problem}

\begin{problem}
    % 2
    Студент выучил $m$ вопросов из $n$ (здесь $m < n$).
    Экзаменатор спросил $k$ вопросов (предположим, что $k \leq m$ и $k \leq n - m$).
    Случайная величина $X$ -- число известных студенту вопросов, среди заданных.
    Найдите $\EE X$ и $\DD X$.
\end{problem}

% \begin{problem}
%     % 5
%     $18$ мальчиков и $9$ девочек усаживаются на один ряд в кинотеатре (в ряду $27$ мест) случайным образом.
%     Случайная величина $X$ -- число пар ``мальчик, девочка'' или ``девочка, мальчик'' рядом (один и тот же человек может входить и в две такие пары).
%     Найдите $\EE X$.
% \end{problem}

% \begin{problem}
%     % 7
%     Найдите математическое ожидание числа повторений грани при бросании четырех игральных костей.
% \end{problem}

\begin{problem}
    % 9
    Первый стрелок попадает в мишень с вероятностью $0.4$, а второй с вероятностью $0.8$.
    Они сделали по $n$ выстрелов каждый.
    Случайная величина $X$ -- разность между числом попаданий второго и первого.
    Найдите $\EE X$ и $\DD X$.
\end{problem}

\begin{problem}
    % 10
    $X \sim Poiss(\lambda)$, найдите $\EE X$.
\end{problem}

\begin{problem}
    % 11
    Совместное распределение пары $(X, Y)$ задано таблицей
    \begin{center}
    \begin{tabular}{c|c|c|c}
        X \ Y & 0 & 1 & 2 \\
        \hline
        -1 & 0.2 & 0.05 & 0.15 \\
        \hline
        0 & 0.05 & 0.1 & 0.05 \\
        \hline
        1 & 0.05 & 0.15 & 0.2 \\
    \end{tabular}
    \end{center}
    Найдите $\EE X$, $\EE Y$, $\EE XY$, $\DD X$, $\DD Y$, $\CV(X, Y)$. 
\end{problem}
\end{document}
