\documentclass[11pt]{article}
\usepackage[
    a4paper,
    bindingoffset=1cm,
    left=1cm,
    right=1.5cm,
    top=2cm,
    bottom=2cm,
    footskip=1cm
]{geometry}

\usepackage[utf8]{inputenc}
\usepackage[russian]{babel}

\usepackage{amsmath}
\usepackage{amsfonts}
\usepackage{amsmath}
\usepackage{amsthm}
\usepackage{mathrsfs}
\usepackage{amssymb}
\usepackage[unicode, pdftex]{hyperref}
\usepackage{enumitem}

\pagestyle{empty}

% colors package
\usepackage{xcolor}

\DeclareMathOperator\PP{\mathbb{P}}
\DeclareMathOperator\EE{\mathbb{E}}
\DeclareMathOperator\DD{\mathbb{D}}
\DeclareMathOperator\II{\mathbb{I}}
\DeclareMathOperator\FF{\mathcal{F}}

\newtheorem{theorem}{Теорема}
\newtheorem{statement}{Утверждение}
\newtheorem{lemma}{Лемма}
\newtheorem{definition}{Определение}
\newtheorem{notification}{Замечание}
\newtheorem{question}{Вопрос}
\newtheorem{example}{Пример}
\newtheorem{problem}{Задача}

\newcommand{\todo}[1]{{\color{red}[TODO: #1]}}
\newcommand{\remark}[1]{{\color{blue}[REMARK: #1]}}


\title{Семинар 1. Комбинаторика. Классическое вероятностное пространство.}
% \author{}
% \date{}
\date{\vspace{-1cm}}

\begin{document}
\maketitle

\section*{Домашнее задание}
\begin{problem}
    Известно, что $\PP(A) = 0.8,\ \PP(AB) = 0.3,\ \PP(B) = 0.5$.
    Какое из событий $A \setminus B,\ B \setminus A$ более вероятно?
\end{problem}

\begin{problem}
    Известно, что $\PP(A) = 3 / 4,\ \PP(B) = 1 / 3$.
    Показать, что $5 / 12 \leq \PP(A \setminus B) \leq 8 / 12$.
\end{problem}

\begin{problem}
    Найти вероятности того, что при броске 10 игральных кубиков выпало хотя бы 2 шестерки.
\end{problem}

\begin{problem}
    Сколько способов покрасить 12 различных комнат, чтобы было 2 зеленых, 8 красных и 2 синих?
\end{problem}

\begin{problem}
    $N$ человек рассаживаются в ряд в случайном порядке.
    Найти вероятность того, что между двумя определенными лицами окажется ровно $k$ человек.
    А если люди рассаживаются за круглый стол?
\end{problem}

% \begin{problem}
%     За круглый стол в случайном порядке рассаживаются $n$ мужчин $n$ женщин.
%     Какова вероятность того, что их можно разбить на $n$ непересекающихся пар разнополых соседей?
% \end{problem}
% 
% \begin{problem}
%     В первом ряду кинотеатра, состоящем из $N$ мест случайным образом рассаживаются $n$ человек.
%     Найти вероятность того, что каждый из них имеет ровно одного соседа.
% \end{problem}

\end{document}


