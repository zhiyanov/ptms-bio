\documentclass[11pt]{article}

%%%% Geometry
\usepackage[
    a4paper,
    bindingoffset=1cm,
    left=1cm,
    right=1.5cm,
    top=2cm,
    bottom=2cm,
    footskip=1cm
]{geometry}

%%%% Standard Packages
\usepackage{graphicx}%
\usepackage{multirow}%
\usepackage{amsmath,amssymb,amsfonts}%
\usepackage{amsthm}%
\usepackage{mathrsfs}%
\usepackage[title]{appendix}%
\usepackage{xcolor}%
\usepackage{textcomp}%
\usepackage{manyfoot}%
\usepackage{booktabs}%
\usepackage{algorithm}%
\usepackage{algorithmicx}%
\usepackage[noend]{algpseudocode}%
\usepackage{listings}%

%%%% Additional Packages
\usepackage{mathtools}%
\usepackage{enumitem}%
\usepackage{xcolor}%
\usepackage[utf8]{inputenc}%
\usepackage[T2A]{fontenc}%
\usepackage[unicode, pdftex]{hyperref}%
\usepackage{hyphenat}%
\usepackage[russian, english]{babel}%

%%%% Graphix
\usepackage[font=small]{caption}%
\usepackage[labelformat=empty, position=top]{subcaption}%
\usepackage[export]{adjustbox}%
\usepackage{placeins}%

%%%% Commands
\newcommand{\todo}[1]{{\color{red}[TODO: #1]}}%
\newcommand{\red}[1]{{\color{red}#1}}%
\newcommand{\green}[1]{{\color{green}#1}}%
\newcommand{\bin}[2]{{C_{#1}^{#2}}}%

%%%% Operators
\DeclareMathOperator\PP{\mathbb{P}}%
\DeclareMathOperator\EE{\mathbb{E}}%
\DeclareMathOperator\DD{\mathbb{D}}%
\DeclareMathOperator\II{\mathbb{I}}%
\DeclareMathOperator\FF{\mathcal{F}}%
\DeclareMathOperator\CV{\mathrm{cov}}%

\DeclarePairedDelimiter\abs{\lvert}{\rvert}%
\DeclarePairedDelimiter\norm{\lVert}{\rVert}%
\makeatletter
\let\oldabs\abs
\def\abs{\@ifstar{\oldabs}{\oldabs*}}
\let\oldnorm\norm
\def\norm{\@ifstar{\oldnorm}{\oldnorm*}}
\makeatother

%%%% Environments
\newtheorem{theorem}{Теорема}
\newtheorem{statement}{Утверждение}
\newtheorem{lemma}{Лемма}
\newtheorem{definition}{Определение}
\newtheorem{notification}{Замечание}
\newtheorem{question}{Вопрос}
\newtheorem{example}{Пример}
\newtheorem{problem}{Задача}

\title{Классическое вероятностное пространство. Урновые схемы.} 
\date{\vspace{-1cm}}

\begin{document}
\maketitle

\section*{Банк задач}
\begin{problem}
    Известно, что $\PP(A) = 0.8,\ \PP(AB) = 0.3,\ \PP(B) = 0.5$.
    Какое из событий $A \setminus B,\ B \setminus A$ более вероятно?
\end{problem}

\begin{problem}
    Известно, что $\PP(A) = 3 / 4,\ \PP(B) = 1 / 3$.
    Показать, что $5 / 12 \leq \PP(A \setminus B) \leq 8 / 12$.
\end{problem}

\begin{problem}
    Найти вероятности того, что при броске 10 игральных кубиков выпало хотя бы 2 шестерки.
\end{problem}

\begin{problem}
    Сколько способов покрасить 12 различных комнат, чтобы было 2 зеленых, 8 красных и 2 синих?
\end{problem}

\begin{problem}
    $N$ человек рассаживаются в ряд в случайном порядке.
    Найти вероятность того, что между двумя определенными лицами окажется ровно $k$ человек.
    А если люди рассаживаются за круглый стол?
\end{problem}

\begin{problem}
    За круглый стол в случайном порядке рассаживаются $n$ мужчин $n$ женщин.
    Какова вероятность того, что их можно разбить на $n$ непересекающихся пар разнополых соседей?
\end{problem}

\begin{problem}
    В первом ряду кинотеатра, состоящем из $N$ мест случайным образом рассаживаются $n$ человек.
    Найти вероятность того, что каждый из них имеет ровно одного соседа.
\end{problem}

\begin{problem}
    Известно, что $\PP(A) = 0.7, \PP(A \setminus B) = 0.4, \PP(B) = 0.4$.
    Найти вероятность того, что произойдет только одно из событий $A$ и $B$.
    Какова вероятность того, что не случится ни одно из событий $A$ и $B$.
    А что если в условии задачи взять $\PP(A \setminus B) = 0.2$?
\end{problem}

\begin{problem}
    Две ладьи ставятся наугад на шахматную доску на различные клетки.
    Опишите вероятностное пространство.
    Какая вероятность того, что ладьи бьют друг друга?
\end{problem}

\begin{problem}
    Какие из вероятностных пространств соответствуют эксперименту случайного выбора трех людей из пяти наугад, без учета порядка?
    \begin{itemize}
        \item
        упорядоченные последовательности длины три из различных чисел от $1$ до $5$, которым приписаны равные вероятности;
        \item
        последовательности длины пять из $2$ нулей и $3$ единиц, которым приписаны равные вероятности;
        \item
        множества из трех элементов из различных чисел от $1$ до $5$, которым приписаны равные вероятности.
    \end{itemize}
\end{problem}

\begin{problem}
    В партии $16$ деталей из которых $4$ бракованных.
    Найти вероятность того, что из $5$ выбранных деталей ровно $2$ бракованных.
\end{problem}

\begin{problem}
    У Анны есть $n$ чайных пар (чашка и блюдце), все разных цветов.
    Она расставляет блюдца на стол в случайном порядке и на каждое ставит чашку (тоже случайным образом).
    Найти вероятность того, что у всех пар цвет чашки и блюдца \textbf{не} совпадет.
\end{problem}

\begin{problem}
    Найти вероятности того, что при броске $10$ игральных кубиков выпало ровно три шестерки.
\end{problem}

\begin{problem}
    У человека $n$ ключей.
    Найти вероятность, что потребуется ровно $k$ попыток, чтобы открыть дверь, если не подошедшие ключи (a) откладываются, (б) не откладываются.
\end{problem}

\begin{problem}
    Известно, что $\PP(A) = 3 / 4, \PP(B) = 1 / 3$.
    Показать, что $5 / 12 \leq \PP(A \setminus B) \leq 8 / 12$.
\end{problem}

\begin{problem}
    Найти вероятности того, что при броске $10$ игральных кубиков выпало хотя бы две шестерки.
\end{problem}

\begin{problem}
    Сколько способов покрасить $12$ различных комнат, чтобы было $2$ зеленых, $8$ красных и $2$ синих?
\end{problem}

\begin{problem}
    В $n$ ящиков раскладывают $k$ неразличимых между собой шариков по очереди, каждый раз выбирая ящик равновероятно.
    Найти вероятность того, что нет пустых ящиков.
\end{problem}

\begin{problem}
    Два игрока играют в безобидную игру (т.е. шансы на выигрыш одинаковы) и они договорились, что тот, кто первым выиграет $6$ партий, получит весь приз.
    Но игра остановилась, когда первый игрок выиграл $4$ партии, а второй –- $3$.
    Как справедливо разделить приз?
    Опишите вероятностное пространство.
\end{problem}

\begin{problem}
    $N$ человек рассаживаются (а) в ряд (б) за круглый стол в случайном порядке.
    Найти вероятность того, что между двумя определенными лицами окажется ровно $k$ человек.
\end{problem}

\begin{problem}
    За круглый стол в случайном порядке рассаживаются $n$ мужчин $n$ женщин.
    Какова вероятность того, что их можно разбить на $n$ непересекающихся пар разнополых соседей?
\end{problem}

\begin{problem}
    В первом ряду кинотеатра, состоящем из $N$ мест случайным образом рассаживаются $n$ человек.
    Найти вероятность того, что каждый из них имеет ровно одного соседа.
\end{problem}
\end{document}


