\documentclass[11pt]{article}
\usepackage[
    a4paper,
    bindingoffset=1cm,
    left=1cm,
    right=1.5cm,
    top=2cm,
    bottom=2cm,
    footskip=1cm
]{geometry}

\usepackage[utf8]{inputenc}
\usepackage[russian]{babel}

\usepackage{amsmath}
\usepackage{amsfonts}
\usepackage{amsmath}
\usepackage{amsthm}
\usepackage{mathrsfs}
\usepackage{amssymb}
\usepackage[unicode, pdftex]{hyperref}
\usepackage{enumitem}

\pagestyle{empty}

% colors package
\usepackage{xcolor}

\DeclareMathOperator\PP{\mathbb{P}}
\DeclareMathOperator\EE{\mathbb{E}}
\DeclareMathOperator\DD{\mathbb{D}}
\DeclareMathOperator\II{\mathbb{I}}
\DeclareMathOperator\FF{\mathcal{F}}

\newtheorem{theorem}{Теорема}
\newtheorem{statement}{Утверждение}
\newtheorem{lemma}{Лемма}
\newtheorem{definition}{Определение}
\newtheorem{notification}{Замечание}
\newtheorem{question}{Вопрос}
\newtheorem{example}{Пример}
\newtheorem{problem}{Задача}
\newtheorem{hproblem}{Задача}
\newtheorem{fproblem}{Задача}

\newcommand{\todo}[1]{{\color{red}[TODO: #1]}}
\newcommand{\remark}[1]{{\color{blue}[REMARK: #1]}}


\title{Семинар 1. Комбинаторика. Классическое вероятностное пространство.}
% \author{}
% \date{}
\date{\vspace{-1cm}}

\begin{document}
\maketitle

\section*{Бонусные задачи}
\begin{fproblem}
    Два игрока играют в безобидную игру (т.е. шансы на выигрыш одинаковы) и они договорились, что тот, кто первым выиграет $6$ партий, получит весь приз.
    Но игра остановилась, когда первый игрок выиграл $4$ партии, а второй –- $3$.
    Как справедливо разделить приз?
    Опишите вероятностное пространство.
\end{fproblem}

\begin{fproblem}
    Из совокупности всех подмножеств множества $\{1, 2, \ldots, N \}$ по схеме выбора с возвращением выбираются множества $A_1, A_2, \ldots, A_n$.
    Найти вероятность события $\{A_1 A_2 \cdots A_n = \emptyset\}$.
\end{fproblem}

\begin{fproblem}
    Из множества монотоных последовательностей длины $n$, составленных из чисел $1, 2, \ldots, N$, случайны образом выбирается одна.
    Найти вероятность того, что она строго монотонна.
\end{fproblem}
\end{document}
