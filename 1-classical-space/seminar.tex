\documentclass[11pt]{article}

%%%% Geometry
\usepackage[
    a4paper,
    bindingoffset=1cm,
    left=1cm,
    right=1.5cm,
    top=2cm,
    bottom=2cm,
    footskip=1cm
]{geometry}

%%%% Standard Packages
\usepackage{graphicx}%
\usepackage{multirow}%
\usepackage{amsmath,amssymb,amsfonts}%
\usepackage{amsthm}%
\usepackage{mathrsfs}%
\usepackage[title]{appendix}%
\usepackage{xcolor}%
\usepackage{textcomp}%
\usepackage{manyfoot}%
\usepackage{booktabs}%
\usepackage{algorithm}%
\usepackage{algorithmicx}%
\usepackage[noend]{algpseudocode}%
\usepackage{listings}%

%%%% Additional Packages
\usepackage{mathtools}%
\usepackage{enumitem}%
\usepackage{xcolor}%
\usepackage[utf8]{inputenc}%
\usepackage[T2A]{fontenc}%
\usepackage[unicode, pdftex]{hyperref}%
\usepackage{hyphenat}%
\usepackage[russian, english]{babel}%

%%%% Graphix
\usepackage[font=small]{caption}%
\usepackage[labelformat=empty, position=top]{subcaption}%
\usepackage[export]{adjustbox}%
\usepackage{placeins}%

%%%% Commands
\newcommand{\todo}[1]{{\color{red}[TODO: #1]}}%
\newcommand{\red}[1]{{\color{red}#1}}%
\newcommand{\green}[1]{{\color{green}#1}}%
\newcommand{\bin}[2]{{C_{#1}^{#2}}}%

%%%% Operators
\DeclareMathOperator\PP{\mathbb{P}}%
\DeclareMathOperator\EE{\mathbb{E}}%
\DeclareMathOperator\DD{\mathbb{D}}%
\DeclareMathOperator\II{\mathbb{I}}%
\DeclareMathOperator\FF{\mathcal{F}}%

\DeclarePairedDelimiter\abs{\lvert}{\rvert}%
\DeclarePairedDelimiter\norm{\lVert}{\rVert}%
\makeatletter
\let\oldabs\abs
\def\abs{\@ifstar{\oldabs}{\oldabs*}}
\let\oldnorm\norm
\def\norm{\@ifstar{\oldnorm}{\oldnorm*}}
\makeatother

%%%% Environments
\newtheorem{theorem}{Теорема}
\newtheorem{statement}{Утверждение}
\newtheorem{lemma}{Лемма}
\newtheorem{definition}{Определение}
\newtheorem{notification}{Замечание}
\newtheorem{question}{Вопрос}
\newtheorem{example}{Пример}
\newtheorem{problem}{Задача}

\title{Классическое вероятностное пространство. Урновые схемы.} 
\date{\vspace{-1cm}}

\begin{document}
\maketitle

\section*{Введение}
\begin{itemize}
    \item
    итуитивное определение вероятности события через повторение эксперимента большое число раз;
    \item
    повторение не всегда возможно организовать: пример с игрой сборной Англии и Франции на Евро2024;
    \item
    решение: в математике вероятность задается заранее, но из интуитивных соображений.
\end{itemize}

\section*{Вероятностное пространство}
\subsection*{Пространство элементарных исходов}
\begin{itemize}
    \item
    с помощью пространства элементарных исходов можно закодировать или математически описать эксперимент;
    \item
    есть эксперимент с $N$ возможными исходами, тогда в качестве пространства элементарных исходов возьмем множество $\Omega_N = \left\{\omega_1, \ldots, \omega_N\right\}$;
    \item
    эксперимент с подбрасыванием монетки один или несколько раз;
    как закодировать выпадение на ребро?
    \item
    как соотносится случайность и пространство элементарных исходов, при чем здесь события: природа c некоторой вероятностью выбирает какой-то элементарный исход;
    \item
    событие в теории вероятностей -- множество каких-то элементарных исходов.
\end{itemize}

\subsection*{Алгебра событий (дополнительный материал)}
\begin{itemize}
    \item
    обычно нас интересует ``суммарная'' вероятность сразу нескольких элементарных исходов -- для этого исходы объединяют в события;
    \item
    получается, что вероятность -- функция, принимающая на вход множество и возвращающая число;
    \item
    алгебра событий $\FF$ -- область определения такой функции, то есть это набор множеств, вероятность которых мы можем измерить;
    элементы $\FF$ -- события;
    \item 
    в случае дискретного (не более чем счетного) пространства элементарных исходов можно взять в качестве $\FF$ множество всех подмножеств $\Omega$;
    \item
    алгебры можно брать разными, но все они обязаны удовлетворять условиям:
    пустое множество и пространство элементарных исходов принадлежат алгебре;
    операции объединения, пересечения и разности (не более чем счетного числа) множеств не выводят из алгебры;
    \item
    пример с алгеброй для подбрасывания одной и двух монеток;
    \item
    почему нужно задавать алгебру и почему недостаточно брать в качестве нее все возможные подмножества пространства элементарных исходов:
    алгебра может получиться слишком обширной, и на ней не получится задать вероятность, удовлетворяющую классическим условиям (\url{https://en.wikipedia.org/wiki/Vitali\_set}).
\end{itemize}

\subsection*{Вероятность}
\begin{itemize}
    \item
    вероятность можно определить, как отображение из множества подмоножеств $\Omega$ (то есть из $\FF$) на отрезок $[0, 1]$, удовлетворяющее условиям (аксиомам):
    \begin{itemize}
        \item
        неотрицательность: $\PP(A) \geq 0$;
        \item
        конечность: $\PP(\Omega) = 1$;
        \item
        аддитивность: 
        \[
            \PP(A + B) = \PP(A) + \PP(B),
        \]
        для непересекающихся $A$ и $B$;
        \item
        сигма-аддитивность (счетная аддитивность):
        \[
            \PP\left( \sum_{i=1}^\infty A_i \right) = \sum_{i=1}^\infty\PP\left( A_i \right),
        \]
        где $A_1, A_2, \ldots$ не пересекаются.
    \end{itemize}
    \item
    в случае, если пространство элементарных исходов дискретно, то вероятность можно сначала задать на элементарных исходах, а потом продолжить ее на все возможные события:
    \[
        \PP(A) = \sum_{\omega \in A} \PP(\omega),\ A \subseteq \Omega;
    \]
    \item
    формулы включения и исключения, следующиe из аксиом:
    \begin{gather*}
        \PP(A \cup B) = \PP(A) + \PP(B) - \PP(A \cap B), \\
        \PP\left( \cup_{i \leq n} A_i \right) = \sum_{i \leq n} \PP\left( A_i \right) - \sum_{i_1 < i_2} \PP\left( A_{i_1} A_{i_2} \right) + \ldots - (-1)^n \PP\left( A_1 \ldots A_n \right).
    \end{gather*}
\end{itemize}

\section*{Классическое вероятностное пространство}
\begin{itemize}
    \item
    так все-таки из каких соображений задавать вероятность?
    \item
    из соображений симметрии можно считать, что все элементарные исходы равновероятны, то есть $\PP(\omega) = 1 / N$ для любого $\omega \in \Omega_N$;
    \item
    пример симметричного и несимметричного пространства исходов при подбрасывании двух монеток/кубиков.    
\end{itemize}

\section*{Урновые схемы}
\begin{itemize}
    \item
    рассмотрим урну с шарами, пронумерованными от $1$ до $n$, мы вытягиваем $k$ шаров и записываем их номера;
    \item
    упорядоченный, с возвращением:
    \[
        \Omega = \left\{ (i_1, \ldots, i_k): 1 \leq i_1, \ldots, i_k \leq n \right\},
    \]
    количество элементарных исходов равно $n^k$;
    \item
    упорядоченный, без возвращения:
    \[
        \Omega = \left\{ (i_1, \ldots, i_k): 1 \leq i_1, \ldots, i_k \leq n,\ i_m \neq i_l \right\},
    \]
    количество элементарных исходов равно $A_n^k$;
    \item
    неупорядоченный, без возвращения: 
    \[
        \Omega = \left\{ \{i_1, \ldots, i_k\}: 1 \leq i_1, \ldots, i_k \leq n,\ i_m \neq i_l \right\},
    \]
    количество элементарных исходов равно $C_n^k$;
    \item
    неупорядоченный, c возвращением: 
    \[
        \Omega = \left\{(i_1, \ldots, i_n): i_1 + \cdots + i_n = k \right\},
    \]
    количество элементарных исходов равно $C^{n - 1}_{n + k - 1}$:
    \begin{itemize}
        \item подсчет количества шаров заданного класса: $i_1$ -- количество шаров типа с номером $1$;
        \item расстановка $n - 1$ перегородки между $k$ черными шарами;
        \item раскраска $n - 1$ шара в белый цвет среди $n + k - 1$.
    \end{itemize} 
\end{itemize}

\section*{Задачи}
\begin{problem}
    Известно, что $\PP(A) = 0.7,\ \PP(A \setminus B) = 0.4,\ \PP(B) = 0.4$.
    Найти вероятность того, что произойдет только одно из событий $A$ или $B$.
    Какова вероятность того, что не случится ни одно из событий $A$ или $B$.
    Можно ли в условии задачи заменить $\PP(A \setminus B)$ на  $0.2$?
\end{problem}

\begin{problem}
    Две ладьи ставятся наугад на шахматную доску на различные клетки.
    Опишите вероятностное пространство.
    Какая вероятность того, что ладьи бьют друг друга?
\end{problem}

\begin{problem}
    В партии $16$ деталей из которых $4$ бракованных.
    Найти вероятность того, что из $5$ выбранных деталей ровно $2$ бракованных.
\end{problem}

\begin{problem}
    Найти вероятности того, что при броске $10$ игральных кубиков выпало ровно три шестерки.
\end{problem}
\begin{problem}
    У человека $n$ ключей.
    Найти вероятность, что потребуется ровно $k$ попыток, чтобы открыть дверь, если не подошедшие ключи (a) откладываются, (б) не откладываются.
\end{problem}

\end{document}


