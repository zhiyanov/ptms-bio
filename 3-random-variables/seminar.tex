\documentclass[11pt]{article}

%%%% Geometry
\usepackage[
    a4paper,
    bindingoffset=1cm,
    left=1cm,
    right=1.5cm,
    top=2cm,
    bottom=2cm,
    footskip=1cm
]{geometry}

%%%% Standard Packages
\usepackage{graphicx}%
\usepackage{multirow}%
\usepackage{amsmath,amssymb,amsfonts}%
\usepackage{amsthm}%
\usepackage{mathrsfs}%
\usepackage[title]{appendix}%
\usepackage{xcolor}%
\usepackage{textcomp}%
\usepackage{manyfoot}%
\usepackage{booktabs}%
\usepackage{algorithm}%
\usepackage{algorithmicx}%
\usepackage[noend]{algpseudocode}%
\usepackage{listings}%

%%%% Additional Packages
\usepackage{mathtools}%
\usepackage{enumitem}%
\usepackage{xcolor}%
\usepackage[utf8]{inputenc}%
\usepackage[T2A]{fontenc}%
\usepackage[unicode, pdftex]{hyperref}%
\usepackage{hyphenat}%
\usepackage[russian, english]{babel}%

%%%% Graphix
\usepackage[font=small]{caption}%
\usepackage[labelformat=empty, position=top]{subcaption}%
\usepackage[export]{adjustbox}%
\usepackage{placeins}%

%%%% Commands
\newcommand{\todo}[1]{{\color{red}[TODO: #1]}}%
\newcommand{\red}[1]{{\color{red}#1}}%
\newcommand{\green}[1]{{\color{green}#1}}%
\newcommand{\bin}[2]{{C_{#1}^{#2}}}%

%%%% Operators
\DeclareMathOperator\PP{\mathbb{P}}%
\DeclareMathOperator\EE{\mathbb{E}}%
\DeclareMathOperator\DD{\mathbb{D}}%
\DeclareMathOperator\II{\mathbb{I}}%
\DeclareMathOperator\FF{\mathcal{F}}%

\DeclarePairedDelimiter\abs{\lvert}{\rvert}%
\DeclarePairedDelimiter\norm{\lVert}{\rVert}%
\makeatletter
\let\oldabs\abs
\def\abs{\@ifstar{\oldabs}{\oldabs*}}
\let\oldnorm\norm
\def\norm{\@ifstar{\oldnorm}{\oldnorm*}}
\makeatother

%%%% Environments
\newtheorem{theorem}{Теорема}
\newtheorem{statement}{Утверждение}
\newtheorem{lemma}{Лемма}
\newtheorem{definition}{Определение}
\newtheorem{notification}{Замечание}
\newtheorem{question}{Вопрос}
\newtheorem{example}{Пример}
\newtheorem{problem}{Задача}

\title{Случайные величины, векторы. Распределение случайных величин}
\date{\vspace{-1cm}}

\begin{document}
\maketitle

\section*{Последовательность независимых испытаний (дополнительно)}
\begin{itemize}
    \item
    каждому эксперименту соответствует свое вероятностное пространство; есть серия независимых экспериментов, как сконструировать пространство, описывающее серию?
    \item
    конструкция такого пространства на примере схемы Бернулли;
    \item
    про общую конструкцию можно почитать по \href{https://math.stackexchange.com/questions/4261460/how-to-define-independence-of-random-variables-from-different-probability-spaces}{ссылке}.
\end{itemize}

\section*{Случайные величины}
\begin{itemize}
    \item
    \textbf{случайная величина} -- \emph{измериомое} относительно борелелевской сигма-алгебры $\mathcal{B}(\mathbb{R})$ отображение из $\Omega$ в $\mathbb{R}$;
    \item
    \textbf{распределение случайной величины} $X$ -- вероятностная мера $\PP_X$, опредленная на измеримом пространстве $(\mathbb{R}, \mathcal{B}(\mathbb{R}))$, удовлетврояющая условию 
    \[
        \PP_X(B) = \PP\left( \left\{\omega: X(\omega) \in B\right\} \right),\ B \in \mathcal{B}(\mathbb{R});
    \]
    \item
    \textbf{функция масс}, как вероятность заданного значения случайной величины
    \[
        \PP_X(x_i) = \PP\left( \left\{\omega: X(\omega) = x_i \right\} \right);
    \]
    \item
    задание функции масс табличкой: пример на одной монетке и схеме Бернулли;
    \item 
    вероятностное пространство $(\mathcal{X} = Im(X), 2^{\mathcal{X}} \cap \mathcal{B}(\mathbb{R}), \PP_X)$, порожденное случайной величиной $X$, позволяет нам забыть про исходное вероятностное пространство;
    \item
    этого пространства не достаточно, если речь идет о вероятности события, определнного двумя случайными величинами: пусть $\Omega = \{1, 2, 3\}, X(\omega) = \omega, Y_1(\omega) = \omega, Y_2(\omega) = (\omega + 1)\ \text{mod}\ 3 + 1$, тогда распределения $X, Y_1, Y_2$ совпадают, но 
    \[
        \PP(X = 1, Y_1 = 1) \neq \PP(X = 1, Y_2 = 1).
    \]
\end{itemize}

\section*{Классические дискретные распределения}
\begin{itemize}
    \item
    бернуллиевское распределение $Bernoulli(p)$: $\mathcal{X} = \{0, 1\}$, $\PP_X(0) = 1 - p$, $\PP_X(1) = p$, соответсвует подбрасыванию одной несимметричной монеты;
    \item
    биномиальное распределение $Binom(n, p)$: $\mathcal{X} = \{0, 1, \ldots, n\}$, $\PP_X(k) = C_n^k p^k (1 - p)^{n - k}$, соотвествует числу решек ($1$) при $n$ бросаниях несимметричной монеты;
    \item
    дискретное равномерное распределение $R\{1, \ldots, N\}$: $\mathcal{X} = \{1, \ldots, N\}$, $\PP_X(k) = 1 / N$, описывает номер шара, извлеченного из урны с $N$ шарами;
    \item
    геометрическое распределение $Geom(p)$: $\mathcal{X} = \mathbb{N}_0$, $\PP_X(k) = (1 - p)^k p$, соответсвует количеству орлов ($0$), выпавших при бросании монеты до первой решки ($1$);
    \item
    пуассоновское распределение $Poiss(\lambda)$: $\mathcal{X} = \mathbb{N}_0$, $\PP_X(k) = \frac{\lambda^k}{k!} e^{-\lambda}$, оно хорошо приближает биномиальное раcпределение $Binom(n, pn)$ с $p n \sim \lambda / n$ при больших $n$;
    \item
    отрицательное биномиальное распределение $NegBinom(r, p)$:  $\mathcal{X} = \mathbb{N}_0$, $\PP_X(k) = C_{r + k - 1}^{r - 1} (1 - p)^k p^r$, соответствует количеству орлов ($0$), выпавших при бросании монеты до $r$-ой решки ($1$);
    \item
    гипергеометрическое распределени $HyperGeom(N, M, n)$: 
    \[
        \mathcal{X} = \{\max(0, n - N + M), \ldots, \min(M, n),\ \PP_X(k) = \frac{C_M^k C_{N - M}^{n - k}}{C_N^n}, 
    \]
    соответствует количеству белых шаров среди $n$ извлеченных шаров из урны с $N$ шарами, из которых $M$ белых.
\end{itemize}

\section*{Случайные вектора}
\begin{itemize}
    \item
    случайный вектор, как отображение из $\Omega$ в $\mathbb{R}^d$;
    \item
    случайный вектор, как набор случайных величин;
    \item
    распределение и функцию масс дискретного случайного вектора можно задать табличкой;
    \item
    независимость случайных величин.
\end{itemize}

\section*{Задачи}
\begin{problem}
    Есть три стрелка.
    Первый попадает в мишень с вероятностью $0.2$, второй $0.3$, третий $0.4$.
    Они вместе выстрелили по мишени (каждый один раз).
    Случайная величина $X$ — число пуль, попавших в мишень.
    Запишите ее функцию масс.
\end{problem}

\begin{problem}
    Игрок подбрасывает пару кубиков, пока не выпадут одновременно две шестерки.
    Случайная величина $X$ -- число подбрасываний.
    Составьте функцию масс.
    Как называется такая случайная величина?
\end{problem}

\begin{problem}
    Первый стрелок попадает в мишень с вероятностью $0.4$, а второй — с вероятностью $0.75$.
    Они стреляют по очереди, начиная с первого, до первого попадания.
    Случайная величина $X$ — число сделанных выстрелов.
    Найдите ее функцию масс.
\end{problem}

\begin{problem}
    Функция масс двумерного случайного вектора имеет вид:
    \begin{center}
    \begin{tabular}{c|c|c|c}
        Y / X & 0 & 1 & 2 \\
        \hline
        -1 & 0.2 & 0.05 & 0.15 \\
        \hline
        0 & 0.05 & 0.1 & 0.05 \\
        \hline
        1 & 0.05 & 0.15 & 0.2 \\
    \end{tabular}
    \end{center}
    Найдите функцию масс $X$, функцию масс $Y$, функцию масс $X + Y$, функцию масс $XY$.
\end{problem}

\begin{problem}
    В ящике три красных, четыре синих и пять зеленых шаров.
    Наугад (без возвращения) достали четыре шара.
    Случайная величина $X$ — число красных шаров в выборке.
    Случайная величина $Y$ — число синих.
    Зависимы ли они?
    Найдите функцию масс случайного вектора $(X, Y)$.
\end{problem}

\begin{problem}
    Случайные величины $X$, $Y$ независимы и $\PP(X = k) = \PP(Y = k) = 2^{-k}, k = 1, 2, \ldots$.
    Найти вероятности $\PP(X \leq k),\ \PP(X = Y),\ \PP(X < Y),\ \PP(\min(X, Y) \leq k)$.
\end{problem}
\end{document}