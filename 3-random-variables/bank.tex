\documentclass[11pt]{article}

%%%% Geometry
\usepackage[
    a4paper,
    bindingoffset=1cm,
    left=1cm,
    right=1.5cm,
    top=2cm,
    bottom=2cm,
    footskip=1cm
]{geometry}

%%%% Standard Packages
\usepackage{graphicx}%
\usepackage{multirow}%
\usepackage{amsmath,amssymb,amsfonts}%
\usepackage{amsthm}%
\usepackage{mathrsfs}%
\usepackage[title]{appendix}%
\usepackage{xcolor}%
\usepackage{textcomp}%
\usepackage{manyfoot}%
\usepackage{booktabs}%
\usepackage{algorithm}%
\usepackage{algorithmicx}%
\usepackage[noend]{algpseudocode}%
\usepackage{listings}%

%%%% Additional Packages
\usepackage{mathtools}%
\usepackage{enumitem}%
\usepackage{xcolor}%
\usepackage[utf8]{inputenc}%
\usepackage[T2A]{fontenc}%
\usepackage[unicode, pdftex]{hyperref}%
\usepackage{hyphenat}%
\usepackage[russian, english]{babel}%

%%%% Graphix
\usepackage[font=small]{caption}%
\usepackage[labelformat=empty, position=top]{subcaption}%
\usepackage[export]{adjustbox}%
\usepackage{placeins}%

%%%% Commands
\newcommand{\todo}[1]{{\color{red}[TODO: #1]}}%
\newcommand{\red}[1]{{\color{red}#1}}%
\newcommand{\green}[1]{{\color{green}#1}}%
\newcommand{\bin}[2]{{C_{#1}^{#2}}}%

%%%% Operators
\DeclareMathOperator\PP{\mathbb{P}}%
\DeclareMathOperator\EE{\mathbb{E}}%
\DeclareMathOperator\DD{\mathbb{D}}%
\DeclareMathOperator\II{\mathbb{I}}%
\DeclareMathOperator\FF{\mathcal{F}}%

\DeclarePairedDelimiter\abs{\lvert}{\rvert}%
\DeclarePairedDelimiter\norm{\lVert}{\rVert}%
\makeatletter
\let\oldabs\abs
\def\abs{\@ifstar{\oldabs}{\oldabs*}}
\let\oldnorm\norm
\def\norm{\@ifstar{\oldnorm}{\oldnorm*}}
\makeatother

%%%% Environments
\newtheorem{theorem}{Теорема}
\newtheorem{statement}{Утверждение}
\newtheorem{lemma}{Лемма}
\newtheorem{definition}{Определение}
\newtheorem{notification}{Замечание}
\newtheorem{question}{Вопрос}
\newtheorem{example}{Пример}
\newtheorem{problem}{Задача}

\title{Случайные величины, векторы. Распределение случайных величин}
\date{\vspace{-1cm}}

\begin{document}
\maketitle

\section*{Банк задач}
\begin{problem}
    Есть три стрелка.
    Первый попадает в мишень с вероятностью $0.2$, второй $0.3$, третий $0.4$.
    Они вместе выстрелили по мишени (каждый один раз).
    Случайная величина $X$ — число пуль, попавших в мишень.
    Запишите ее функцию масс.
\end{problem}

\begin{problem}
    Игрок подбрасывает пару кубиков, пока не выпадут одновременно две шестерки.
    Случайная величина $X$ -- число подбрасываний.
    Составьте функцию масс.
    Как называется такая случайная величина?
\end{problem}

\begin{problem}
    Первый стрелок попадает в мишень с вероятностью $0.4$, а второй — с вероятностью $0.75$.
    Они стреляют по очереди, начиная с первого, до первого попадания.
    Случайная величина $X$ — число сделанных выстрелов.
    Найдите ее функцию масс.
\end{problem}

\begin{problem}
    Функция масс двумерного случайного вектора имеет вид:
    \begin{center}
    \begin{tabular}{c|c|c|c}
        Y / X & 0 & 1 & 2 \\
        \hline
        -1 & 0.2 & 0.05 & 0.15 \\
        \hline
        0 & 0.05 & 0.1 & 0.05 \\
        \hline
        1 & 0.05 & 0.15 & 0.2 \\
    \end{tabular}
    \end{center}
    Найдите функцию масс $X$, функцию масс $Y$, функцию масс $X + Y$, функцию масс $XY$.
\end{problem}

\begin{problem}
    В ящике три красных, четыре синих и пять зеленых шаров.
    Наугад (без возвращения) достали четыре шара.
    Случайная величина $X$ — число красных шаров в выборке.
    Случайная величина $Y$ — число синих.
    Зависимы ли они?
    Найдите функцию масс случайного вектора $(X, Y)$.
\end{problem}

\begin{problem}
    Случайные величины $X$, $Y$ независимы и $\PP(X = k) = \PP(Y = k) = 2^{-k}, k = 1, 2, \ldots$.
    Найти вероятности $\PP(X \leq k),\ \PP(X = Y),\ \PP(X < Y),\ \PP(\min(X, Y) \leq k)$.
\end{problem}

\begin{problem}
    У меня в кармане две монеты достоинством $10$ рублей и три монеты по $2$ рубля.
    Я достаю две монеты на ощупь (считаем, что на ощупь монеты неразличимы).
    Случайная величина $X$ -- сумма денег, которую я достал.
    Запишите ее функцию масс.
\end{problem}

\begin{problem}
    На пространстве $\Omega = \{1, 2, \ldots, 8\}$, $\PP(\omega_i) = 1 / 8$ заданы случайные величины $X$ и $Y$:
    \begin{center}
    \begin{tabular}{c|c|c|c|c|c|c|c|c}
        $\omega$ & 1 & 2 & 3 & 4 & 5 & 6 & 7 & 8 \\
        \hline
        X & 2 & 1 & 4 & 4 & 2 & 2 & 1 & 1 \\
        \hline
        Y & 1 & 1 & 0 & 2 & 2 & 1 & 0 & 1 \\
    \end{tabular}
    \end{center}
    Найти а) распределения этих случайных величин, б) распределение вектора $(X, Y)$ (составить табличку).
\end{problem}

\begin{problem}
    По окружности расставили $2$ единицы и $N - 2$ нолика в случайном порядке.
    Случайная величина $X$ — число ноликов между парой единиц (минимальное среди двух дуг).
    Найдите ее функцию масс.
\end{problem}

\begin{problem}
    Случайная величина $X$ не постоянна и не равномерна.
    Привидите пример такой случайной величины, что случайные величины $\sin(X)$ и $\cos(X)$ иметь одинаковые функции масс?
\end{problem}

\begin{problem}
    Совместное распределение случайных величин $\xi, \eta$, имеет вид 
    \[
        \PP(\xi = x, \eta = y) = \frac{C (x + y) a^{x+y}}{x! \ y!}, \ x, y \geq 0
    \]
    Найти а) константу C, б) распределение (функцию масс) с.в. $\xi + \eta$.
\end{problem}

\begin{problem}
    Найти распределение (функцию масс) суммы $X_1 + X_2$, где а) $X_i \sim Geom(p)$, б) $X_i \sim Poiss(\lambda_i)$, $X_i$ независимы.\\    
\end{problem}

\begin{problem}
    Найти распределение (функцию масс) суммы $X_1 + X_2$, где $X_i$ н.о.р. из распределения $Poiss(\lambda_i)$
\end{problem}

\begin{problem}
    Брошено $N$ одинаковых несимметричных монет, затем монеты, которые упали орлом вверх, бросают
еще раз.
    Найти распределение итогового количества решек (которое получилось в результате процедуры из двух подбрасываний).
\end{problem}

\begin{problem}
    Найти распределение суммы $X_1 + \cdots + X_n$, где $X_i$ -- н.о.р. а) $X_i \sim Geom(p)$, б) $X_i \sim Poiss(\lambda_i)$.
\end{problem}

\begin{problem}
    Найти распределение суммы случайного числа слагаемых $X_1 + \cdots \ + X_{\eta}$, если слагаемые н.о.р. и $X_i \sim Poiss(\lambda_i)$, $\eta \sim Geom(p)$.
\end{problem}

\begin{problem}
    $X_1, X_2$ -- н.о.р. $Geom(p_i)$ случайные величины.
    Исследуйте на независимость случайные величины $\min(X_1, X_2)$ и $X_1 - X_2$.
\end{problem}
\end{document}