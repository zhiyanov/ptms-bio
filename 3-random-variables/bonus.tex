\documentclass[11pt]{article}

%%%% Geometry
\usepackage[
    a4paper,
    bindingoffset=1cm,
    left=1cm,
    right=1.5cm,
    top=2cm,
    bottom=2cm,
    footskip=1cm
]{geometry}

%%%% Standard Packages
\usepackage{graphicx}%
\usepackage{multirow}%
\usepackage{amsmath,amssymb,amsfonts}%
\usepackage{amsthm}%
\usepackage{mathrsfs}%
\usepackage[title]{appendix}%
\usepackage{xcolor}%
\usepackage{textcomp}%
\usepackage{manyfoot}%
\usepackage{booktabs}%
\usepackage{algorithm}%
\usepackage{algorithmicx}%
\usepackage[noend]{algpseudocode}%
\usepackage{listings}%

%%%% Additional Packages
\usepackage{mathtools}%
\usepackage{enumitem}%
\usepackage{xcolor}%
\usepackage[utf8]{inputenc}%
\usepackage[T2A]{fontenc}%
\usepackage[unicode, pdftex]{hyperref}%
\usepackage{hyphenat}%
\usepackage[russian, english]{babel}%

%%%% Graphix
\usepackage[font=small]{caption}%
\usepackage[labelformat=empty, position=top]{subcaption}%
\usepackage[export]{adjustbox}%
\usepackage{placeins}%

%%%% Commands
\newcommand{\todo}[1]{{\color{red}[TODO: #1]}}%
\newcommand{\red}[1]{{\color{red}#1}}%
\newcommand{\green}[1]{{\color{green}#1}}%
\newcommand{\bin}[2]{{C_{#1}^{#2}}}%

%%%% Operators
\DeclareMathOperator\PP{\mathbb{P}}%
\DeclareMathOperator\EE{\mathbb{E}}%
\DeclareMathOperator\DD{\mathbb{D}}%
\DeclareMathOperator\II{\mathbb{I}}%
\DeclareMathOperator\FF{\mathcal{F}}%

\DeclarePairedDelimiter\abs{\lvert}{\rvert}%
\DeclarePairedDelimiter\norm{\lVert}{\rVert}%
\makeatletter
\let\oldabs\abs
\def\abs{\@ifstar{\oldabs}{\oldabs*}}
\let\oldnorm\norm
\def\norm{\@ifstar{\oldnorm}{\oldnorm*}}
\makeatother

%%%% Environments
\newtheorem{theorem}{Теорема}
\newtheorem{statement}{Утверждение}
\newtheorem{lemma}{Лемма}
\newtheorem{definition}{Определение}
\newtheorem{notification}{Замечание}
\newtheorem{question}{Вопрос}
\newtheorem{example}{Пример}
\newtheorem{problem}{Задача}

\title{Случайные величины, векторы. Распределение случайных величин}
\date{\vspace{-1cm}}

\begin{document}
\maketitle

\section*{Бонусные задачи}
\begin{problem}
    Брошено $N$ одинаковых несимметричных монет, затем монеты, которые упали орлом вверх, бросают
еще раз.
    Найти распределение итогового количества решек (которое получилось в результате процедуры из двух подбрасываний).
\end{problem}

% \begin{problem}
%     Найти распределение суммы $X_1 + \cdots + X_n$, где $X_i$ -- н.о.р. а) $X_i \sim Geom(p)$, б) $X_i \sim Poiss(\lambda_i)$.
% \end{problem}

% \begin{problem}
%     Найти распределение суммы случайного числа слагаемых $X_1 + \cdots \ + X_{\eta}$, если слагаемые н.о.р. и $X_i \sim Poiss(\lambda_i)$, $\eta \sim Geom(p)$.
% \end{problem}

\begin{problem}
    $X_1, X_2$ -- н.о.р. $Geom(p_i)$ случайные величины.
    Исследуйте на независимость случайные величины $\min(X_1, X_2)$ и $X_1 - X_2$.
\end{problem}
\end{document}


