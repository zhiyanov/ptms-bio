\documentclass[11pt]{article}

%%%% Geometry
\usepackage[
    a4paper,
    bindingoffset=1cm,
    left=1cm,
    right=1.5cm,
    top=2cm,
    bottom=2cm,
    footskip=1cm
]{geometry}

%%%% Standard Packages
\usepackage{graphicx}%
\usepackage{multirow}%
\usepackage{amsmath,amssymb,amsfonts}%
\usepackage{amsthm}%
\usepackage{mathrsfs}%
\usepackage[title]{appendix}%
\usepackage{xcolor}%
\usepackage{textcomp}%
\usepackage{manyfoot}%
\usepackage{booktabs}%
\usepackage{algorithm}%
\usepackage{algorithmicx}%
\usepackage[noend]{algpseudocode}%
\usepackage{listings}%

%%%% Additional Packages
\usepackage{mathtools}%
\usepackage{enumitem}%
\usepackage{xcolor}%
\usepackage[utf8]{inputenc}%
\usepackage[T2A]{fontenc}%
\usepackage[unicode, pdftex]{hyperref}%
\usepackage{hyphenat}%
\usepackage[russian, english]{babel}%

%%%% Graphix
\usepackage[font=small]{caption}%
\usepackage[labelformat=empty, position=top]{subcaption}%
\usepackage[export]{adjustbox}%
\usepackage{placeins}%

%%%% Commands
\newcommand{\todo}[1]{{\color{red}[TODO: #1]}}%
\newcommand{\red}[1]{{\color{red}#1}}%
\newcommand{\green}[1]{{\color{green}#1}}%
\newcommand{\bin}[2]{{C_{#1}^{#2}}}%

%%%% Operators
\DeclareMathOperator\PP{\mathbb{P}}%
\DeclareMathOperator\EE{\mathbb{E}}%
\DeclareMathOperator\DD{\mathbb{D}}%
\DeclareMathOperator\II{\mathbb{I}}%
\DeclareMathOperator\FF{\mathcal{F}}%
\DeclareMathOperator\CV{\mathrm{cov}}%

\DeclarePairedDelimiter\abs{\lvert}{\rvert}%
\DeclarePairedDelimiter\norm{\lVert}{\rVert}%
\makeatletter
\let\oldabs\abs
\def\abs{\@ifstar{\oldabs}{\oldabs*}}
\let\oldnorm\norm
\def\norm{\@ifstar{\oldnorm}{\oldnorm*}}
\makeatother

%%%% Environments
\newtheorem{theorem}{Теорема}
\newtheorem{statement}{Утверждение}
\newtheorem{lemma}{Лемма}
\newtheorem{definition}{Определение}
\newtheorem{notification}{Замечание}
\newtheorem{question}{Вопрос}
\newtheorem{example}{Пример}
\newtheorem{problem}{Задача}

\title{Случайные величины, векторы. Распределение случайных величин}
\date{\vspace{-1cm}}

\begin{document}
\maketitle

\section*{Домашнее задание}
\begin{problem}
    У меня в кармане две монеты достоинством $10$ рублей и три монеты по $2$ рубля.
    Я достаю две монеты на ощупь (считаем, что на ощупь монеты неразличимы).
    Случайная величина $X$ -- сумма денег, которую я достал.
    Запишите ее функцию масс.
\end{problem}

\begin{problem}
    На пространстве $\Omega = \{1, 2, \ldots, 8\}$, $\PP(\omega_i) = 1 / 8$ заданы случайные величины $X$ и $Y$:
    \begin{center}
    \begin{tabular}{c|c|c|c|c|c|c|c|c}
        $\omega$ & 1 & 2 & 3 & 4 & 5 & 6 & 7 & 8 \\
        \hline
        X & 2 & 1 & 4 & 4 & 2 & 2 & 1 & 1 \\
        \hline
        Y & 1 & 1 & 0 & 2 & 2 & 1 & 0 & 1 \\
    \end{tabular}
    \end{center}
    Найти а) распределения этих случайных величин, б) распределение вектора $(X, Y)$ (составить табличку).
\end{problem}

\begin{problem}
    По окружности расставили $2$ единицы и $N - 2$ нолика в случайном порядке.
    Случайная величина $X$ — число ноликов между парой единиц (минимальное среди двух дуг).
    Найдите ее функцию масс.
\end{problem}

\begin{problem}
    Случайная величина $X$ не постоянна и не равномерна.
    Привидите пример такой случайной величины, что случайные величины $\sin(X)$ и $\cos(X)$ иметь одинаковые функции масс?
\end{problem}

% \begin{problem}
%     Совместное распределение случайных величин $\xi, \eta$, имеет вид 
%     \[
%         \PP(\xi = x, \eta = y) = \frac{C (x + y) a^{x+y}}{x! \ y!}, \ x, y \geq 0
%     \]
%     Найти а) константу C, б) распределение (функцию масс) с.в. $\xi + \eta$.
% \end{problem}

% \begin{problem}
%     Найти распределение (функцию масс) суммы $X_1 + X_2$, где а) $X_i \sim Geom(p)$, б) $X_i \sim Poiss(\lambda_i)$, $X_i$ независимы.\\    
% \end{problem}

\begin{problem}
    Найти распределение (функцию масс) суммы $X_1 + X_2$, где $X_i$ н.о.р. из распределения $Poiss(\lambda_i)$.
\end{problem}
\end{document}
